\documentclass[a4paper,12pt,oneside]{article}

\usepackage[textwidth=400pt,textheight=730pt,top=60pt,left=100pt]{geometry}
\usepackage[english]{babel}
\usepackage{hyperref}
\usepackage[nolist]{acronym}
%\usepackage[onehalfspacing]{setspace}
\usepackage{setspace}
\usepackage{booktabs}
\usepackage{graphicx}
\usepackage{float}
\usepackage{amsmath}
\usepackage{listings}
\usepackage{cite}
\usepackage{tikz}
\usepackage{wasysym}
\usepackage{amssymb}
\usepackage{dirtree}
\usepackage[figuresright]{rotating}
\usepackage{subcaption}
\usepackage{adjustbox}
\usepackage{lineno}
\usepackage{multicol}
\linenumbers

\usetikzlibrary{matrix,calc}
%---------------------------
\begin{document}
%
\noindent
%
\LARGE
\textbf{Comparing uncertainties for effective area}
\normalsize\\
%
\begin{center}
Sebastian Achim Mueller, 2020\,March\,27
\end{center}
%
We conclude that the effective area in our grid-based estimate for the Cherenkov-plenoscope is
%
\begin{eqnarray*}
A_\text{eff} &=& \frac{A_G}{M\,N_G} \sum_m^M q_m\,N_{S,m}\\
&=&\frac{A_G}{M\,N_G} \sum_\text{acc.}N_{S,m},
\end{eqnarray*}
%
in an individual energy-bin.
%
Here $A_G$ is the area of the entire grid, $N_G$ is the number of all grid-bins, $M$ is the number of thrown air-showers, and $N_{S,m}$ is the $m$-th air-shower's number of grid-bins to pass a lose threshold of Cherenkov-photons in the field-of-view.
%
The sum $\sum_\text{acc.}^Q$ runs only over the $Q$ accepted of the $M$ thrown air-showers with $q_m = 1$.
%
\subsection*{Uncertainties}
%
Sebastian's estimate for $\Delta A_\text{eff}$ neglects the grid.
%
It assumes that all uncertainty comes from the counting-uncertainty $\Delta Q = \sqrt{Q}$ of the number of air-showers $Q$ that triggered the plenoscope in the detailed simulation.
%
In the source-code, Sebastian estimates $Q$ with
%
\begin{eqnarray*}
Q &=& \sum_m^M q_m.
\end{eqnarray*}
%
To express $A_\text{eff}$ with $Q$ here, Sebastian approximates
%
\begin{eqnarray*}
\sum_\text{acc}^Q N_{S,m} &\approx& Q\,\langle N_S \rangle,
\end{eqnarray*}
%
so that the effective area is
%
\begin{eqnarray*}
A_\text{eff} &\approx& \frac{A_G}{M\,N_G} Q \langle N_{S} \rangle.
\end{eqnarray*}
%
Using
%
\begin{eqnarray*}
\Delta A_\text{eff} &=& \sqrt{\left(\frac{\partial A_\text{eff}}{\partial Q}\right)^2(\Delta Q)^2},
\end{eqnarray*}
%
Sebastian finds
%
\begin{eqnarray*}
\Delta A_\text{eff} &=& \frac{A_G}{N_G\,M} \langle N_S \rangle \sqrt{Q}\\
%
\frac{\Delta A_\text{eff}}{A_\text{eff}} &=& \frac{1}{\sqrt{Q}}
\end{eqnarray*}
%
Werner on the other hand assumes the uncertainty to arise from the counting-uncertainty  $\Delta N_{S,m}$ of the $N_{S,m}$ grid-bins above the lose threshold.
%
Here is how I understood Werner's estimate.
%
He assumes all the gird-bins of the $m$-th air-shower are correlated by 100\% so that
%
\begin{eqnarray*}
\Delta N_{S,m} &=& N_{S,m}.
\end{eqnarray*}
%
Using the propagated uncertainty for uncorrelated air-showers $m$
%
\begin{eqnarray*}
\Delta A_\text{eff} &=& \sqrt{\sum_\text{acc.} \left(\frac{\partial A_\text{eff}}{\partial N_{S,m}}\right)^2(\Delta N_{S,m})^2},
\end{eqnarray*}
%
yields
%
\begin{eqnarray*}
\Delta A_\text{eff} &=& \frac{A_G}{N_G\,M} \sqrt{\sum_\text{acc.}N_{S,m}^2},
\end{eqnarray*}
%
and
%
\begin{eqnarray*}
\frac{\Delta A_\text{eff}}{A_\text{eff}} &=& \frac{\sqrt{\sum_\text{acc.}N_{S,m}^2}}{\sum_\text{acc.}N_{S,m}}.
\end{eqnarray*}
%
When we approximate
%
\begin{eqnarray*}
\sum_\text{acc.}N_{S,m}^2 &\approx& Q \langle N_S^2 \rangle,
\end{eqnarray*}
%
the ratio of the two estimates is
%
\begin{eqnarray*}
\frac{%
\Delta A_\text{eff}^\text{Werner}%
}{%
\Delta A_\text{eff}^\text{Sebastian}}%
%
&\approx&
\frac{%
\sqrt{\langle N_S^2 \rangle}
}{%
\langle N_S\rangle
}
\end{eqnarray*}
%
%
%\subsection*{Questioning the grid}
%%
%We created the grid to estimate the plenoscope-response more efficiently for low %energetic cosmic-rays that otherwise require huge scatter-areas when their air-showers %are deflected in earth's magnetic field.
%%
%But the grid adds complexity to the source-code and to the estimate for the uncertainty.
%%
%I ask myself, if our estimate was simpler when we removed the grid, and used our %knowledge about the cosmic-ray's deflection instead.\\
%%
%We would not longer use the primary particle's core-position as the center for our areal %scatter, but the typical position of the Cherenkov-pool instead.
%%
%Of course, estimating the deflection of cosmic-rays also adds complexity, but at least %this is a rather isolated task.
%%
%One challenge here will be that the areal distribution of the Cherenkov-pool can have %very asymmetric shapes that might need to be corrected for to not lose too much %efficiency.
%
%
\end{document}
%